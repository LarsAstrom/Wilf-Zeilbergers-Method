%%%%%%%%%%%%%%%%%%%%%%%%%%%%%%%%%%%%%%%%%
% Beamer Presentation
% LaTeX Template
% Version 1.0 (10/11/12)
%
% This template has been downloaded from:
% http://www.LaTeXTemplates.com
%
% License:
% CC BY-NC-SA 3.0 (http://creativecommons.org/licenses/by-nc-sa/3.0/)
%
%%%%%%%%%%%%%%%%%%%%%%%%%%%%%%%%%%%%%%%%%

%----------------------------------------------------------------------------------------
%	PACKAGES AND THEMES
%----------------------------------------------------------------------------------------

\documentclass{beamer}

\mode<presentation> {

% The Beamer class comes with a number of default slide themes
% which change the colors and layouts of slides. Below this is a list
% of all the themes, uncomment each in turn to see what they look like.

%\usetheme{default}
%\usetheme{AnnArbor}
%\usetheme{Antibes} %maybe
%\usetheme{Bergen}
%\usetheme{Berkeley}
%\usetheme{Berlin} %maybe+
%\usetheme{Boadilla}
%\usetheme{CambridgeUS}
%\usetheme{Copenhagen}
%\usetheme{Dresden} %maybe+
%\usetheme{Goettingen}
%\usetheme{Hannover}
\usetheme{Ilmenau} %maybe+
%\usetheme{Luebeck}
%\usetheme{Madrid}
%\usetheme{Malmoe}
%\usetheme{Marburg}
%\usetheme{Montpellier}
%\usetheme{PaloAlto}
%\usetheme{Pittsburgh}
%\usetheme{Rochester}
%\usetheme{Singapore}
%\usetheme{Szeged}
%\usetheme{Warsaw}

% As well as themes, the Beamer class has a number of color themes
% for any slide theme. Uncomment each of these in turn to see how it
% changes the colors of your current slide theme.

%\usecolortheme{albatross}
%\usecolortheme{beaver}
%\usecolortheme{beetle}
%\usecolortheme{crane}
%\usecolortheme{dolphin}
%\usecolortheme{dove}
%\usecolortheme{fly}
%\usecolortheme{lily}
%\usecolortheme{orchid}
%\usecolortheme{rose}
%\usecolortheme{seagull}
%\usecolortheme{seahorse}
%\usecolortheme{whale}
%\usecolortheme{wolverine}

%\setbeamertemplate{footline} % To remove the footer line in all slides uncomment this line
%\setbeamertemplate{footline}[page number] % To replace the footer line in all slides with a simple slide count uncomment this line

%\setbeamertemplate{navigation symbols}{} % To remove the navigation symbols from the bottom of all slides uncomment this line
}

% Transparent for next thing
\setbeamercovered{transparent}

\setbeamertemplate{page number in head/foot}{\insertframenumber /\inserttotalframenumber}

\setbeamersize{text margin left=5mm,text margin right=5mm}
\addtobeamertemplate{frametitle}{}{\vspace{-5mm}}


%\addtolength{\headsep}{-1.0cm}

\let\oldforall\forall
\renewcommand{\forall}{\hspace*{2mm}\oldforall\hspace*{1mm}}
\newcommand{\ubf}[1]{\underline{\textbf{#1}}}

\usepackage{graphicx} % Allows including images
\usepackage{booktabs} % Allows the use of \toprule, \midrule and \bottomrule in tables
\usepackage{ragged2e}

\usepackage{appendixnumberbeamer}

%----------------------------------------------------------------------------------------
%	TITLE PAGE
%----------------------------------------------------------------------------------------

\title[Polynomial Computer Algebra and implmentation of Wilf-Zeilberger's method]{Polynomial Computer Algebra and implmentation of Wilf-Zeilberger's method} % The short title appears at the bottom of every slide, the full title is only on the title page

\author{Lars Åström} % Your name
\institute[LTH] % Your institution as it will appear on the bottom of every slide, may be shorthand to save space
{
Supervisor: Victor Ufnarovski \\
\medskip
Faculty of Engineering at Lund University \\ % Your institution for the title page
\medskip
\textit{lars96astrom@gmail.com} % Your email address
}
\date{December 19, 2019} % Date, can be changed to a custom date

\begin{document}

\setlength{\abovedisplayskip}{2pt}
\setlength{\belowdisplayskip}{2pt}

\begin{frame}
\titlepage % Print the title page as the first slide
\end{frame}

\begin{frame}
\frametitle{Overview} % Table of contents slide, comment this block out to remove it
\tableofcontents % Throughout your presentation, if you choose to use \section{} and \subsection{} commands, these will automatically be printed on this slide as an overview of your presentation
\end{frame}

%----------------------------------------------------------------------------------------
%	PRESENTATION SLIDES
%----------------------------------------------------------------------------------------

%------------------------------------------------
\section{Introduction}
%------------------------------------------------
\begin{frame}
  \frametitle{What is the thesis about?}
  %\tableofcontents[currentsection]
  \Large
  Polynomial Computer Algebra
  \pause
  and implementation of Wilf-Zeilberger's method
\end{frame}

\begin{frame}
  \frametitle{What and Why?}
  \begin{columns}[t]

  \column{.5\textwidth} % Left column and width
  \centering
  \Large\ubf{WHAT}\normalsize

  \justify
  Show that $$\sum_{k=n}^\infty \frac{1}{\binom{k}{n}}=\frac{n}{n-1}.$$
  \pause
  \textbf{Characteristics}
  \begin{itemize}
    \item Summation on one side.
    \item Show that...
    \item Often binomial coefficients.
  \end{itemize}
  \pause
  \column{.5\textwidth} % Right column and width
  \centering
  \Large\ubf{WHY}\normalsize

  \justify
  \begin{itemize}
    \item Wilf-Zeilberger's method $\rightarrow$ not so much
    \pause
    \item Automized proof generation $\rightarrow$ a lot
    \pause
    \item Computer Algebra $\rightarrow$ a lot
  \end{itemize}
  \end{columns}
\end{frame}

\begin{frame}
  \frametitle{Short version of the thesis}
  \begin{columns}[c]
    \column{.4\textwidth}
    \begin{itemize}
      \item<1> Historical background
      \item<2> Polynomials
      \item<3> Wilf-Zeilberger's method
      \item<4> Gosper's algorithm
      \item<5> Results
      \item<6> Conclusions
    \end{itemize}
    \column{.6\textwidth}
    \setbeamercovered{invisible}
    \only<1>{
    \center
    \ubf{Important findings}

    \justify
    \begin{itemize}
      \item 1960s: Computer Algebra
      \item 1978: Gosper's Algorithm
      \item 1990: Wilf-Zeilberger's method
      \item 1994: WZ implemented in Mathematica
    \end{itemize}
    }
    \only<2>{
    \begin{itemize}
      \item Used for implementation of WZ
      \item Polynomial $$p(k)=a_0+a_1k+\ldots+a_mk^m$$ is stored as $$[a_0,a_1,\ldots,a_m]$$
      \item Coefficients $a_i$ can be integers or polynomials
    \end{itemize}
    }
    \only<3>{
    \begin{itemize}
      \item Used to prove identities on the form $$\sum_k F(n,k)=1$$
      \item Does this by proving $$\sum_k F(n+1,k)=\sum_k F(n,k)$$
      \item Which is done by ''changing variables''
    \end{itemize}
    }
    \only<4>{
    \begin{itemize}
      \item An algorithm to find a function $S$ such that $$a_k=S_k-S_{k-1}$$
      \item Finds the change of variables needed in WZ
    \end{itemize}
    }
    \only<5>{
    \begin{itemize}
      \item The program writes formal proofs
      \item Proves 80\% of the examples
      \item The remaining seem impossible to prove by WZ method
    \end{itemize}
    }
    \only<6>{
    \begin{itemize}
      \item The program seems to work well, although cannot solve all examples
      \item Computer Algebra quickly gets complicated
    \end{itemize}
    }
  \end{columns}
\end{frame}

\section[Wilf-Zeilberger's method]{Wilf-Zeilberger's method (Wilf, 1990)}
\begin{frame}
  \frametitle{What problems can be solved?}
  Problems on the form
  $$\sum_k F(n,k)=1$$
  can be solved.
  \pause
  Problems on the form
  $$\sum_k A(n,k)=B(n)$$
  can get converted to the right form.
\end{frame}
\begin{frame}
  \frametitle{The idea}
  Want to prove
  \begin{equation*}\label{Eq: WZ1}
    \sum_k F(n,k)=1
  \end{equation*}
  \pause
  Find $G(n,k)$ such that $F(n+1,k)-F(n,k)=G(n,k+1)-G(n,k)$ and $\lim_{k\to\pm\infty} G(n,k)=0$.\pause Now $$\sum_k F(n+1,k)-F(n,k)=\sum_k G(n,k+1)-G(n,k)=0.$$\pause Therefore $$\sum_k F(n,k)$$ is constant for all $n$, therefore if we can evaluate for one $n$ then we are done.
\end{frame}
\begin{frame}
  \frametitle{Steps of the method}
  \begin{columns}[t]

  \column{.5\textwidth} % Left column and width
  \begin{enumerate}
    \item<1-> Start with $$\sum_k A(n,k)=B(n)$$
    \item<3-> Let $F(n,k)=\frac{A(n,k)}{B(n)}$
    \item<5-> Find $G(n,k)$ such that the conditions are satisfied
    \item<7-> Show that $\sum_k F(n^\prime,k)=1$ for some $n^\prime$
  \end{enumerate}

  \column{.5\textwidth} % Right column and width
  \begin{enumerate}
    \item<2-> We have $$\sum_k \binom{n}{k}=2^n$$
    \item<4-> $F(n,k)=\frac{\binom{n}{k}}{2^n}$
    \item<6-> Let $G(n,k)=-\frac{\binom{n}{k-1}}{2^{n+1}}$. Now the conditions are satisfied.
    \item<8-> For $n=0$ we have $\sum_k F(n,k)=\frac{\binom{0}{0}}{2^0}=1,$ thus we have proved the identity.
  \end{enumerate}
  \end{columns}
\end{frame}

\section[Gosper's algorithm]{Gosper's algorithm (Gosper, 1978)}
\begin{frame}
  \frametitle{What problems can be solved?}
  Given a polynomial $a_k$, Gosper's algorithm finds a polynomial $S_k$ such that $$a_k=S_k-S_{k-1}.$$
  \pause
  With $a_k=F(n+1,k)-F(n,k)$ we get that $G(n,k)=S_{k-1}$ makes the first condition in Wilf-Zeilberger's method fulfilled.
\end{frame}
\begin{frame}
  \frametitle{Steps of the algorithm}
  \begin{columns}[t]

  \column{.35\textwidth} % Left column and width
  \begin{enumerate}
    \item<1-> Find polynomials $p_k,q_k,r_k$ such that $gcd(q_k,r_{k+j})=1$ $\forall j\geq 0$ and $$\frac{a_k}{a_{k-1}}=\frac{p_k}{p_{k-1}}\frac{q_k}{r_k}$$
    \item<2-> Find polynomial $f_k$ such that $p_k=q_{k+1}f_k-r_kf_{k-1}$
    \item<3-> Let $S_k=\frac{q_{k+1}}{p_k}f_ka_k$
  \end{enumerate}

  \column{.65\textwidth} % Right column and width
  \only<4>{
  Now we see that
  \begin{equation*}
    \begin{split}
      S_k-S_{k-1} = \frac{q_{k+1}}{p_k}f_ka_k-\frac{q_k}{p_{k-1}}f_{k-1}a_{k-1} &=\\
       = \frac{a_k}{p_k}\Big(q_{k+1}f_k-\frac{q_k}{p_{k-1}}f_{k-1}p_k\frac{a_{k-1}}{a_k}\Big) &= \\
       =\frac{a_k}{p_k}\Big(q_{k+1}f_k-\frac{q_k}{p_{k-1}}f_{k-1}p_k\frac{p_{k-1}}{p_k}\frac{r_k}{q_k}\Big)&= \\
       = \frac{a_k}{p_k}\Big(q_{k+1}f_k-r_kf_{k-1}\Big) = \frac{a_k}{p_k}p_k &= a_k,
    \end{split}
  \end{equation*}
  which means that this $S_k$ indeed is a solution.
  }
  \only<5-7>{
  \begin{enumerate}
    \item<5-> For $\sum_k \binom{n}{k}=2^n$ we get $$\frac{a_k}{a_{k-1}}=\frac{(2k-n-1)(n+2-k)}{k(2k-n-3)}$$ which gives us $p_k=2k-n-1$, $q_k=n+2-k$, $r_k=k$.
    \item<6-> In $$2k-n-1=(n+1-k)f_k-kf_{k-1}$$ we see that $f_k=-1$ gives a solution.
    \item<7> Now we get $S_k=-\frac{n+1-k}{2k-n-1}a_k=-\frac{\binom{n}{k}}{2^{n+1}}$, which corresponds to the $G(n,k)$ we got in the previous example.
  \end{enumerate}
  }
  \end{columns}
\end{frame}

\section{Implementation}
\begin{frame}
  \textbf{2300 lines of code}
  \pause
  \begin{itemize}
    \item 50\% methods for polynomials and Wilf-Zeilberger's method
    \pause
    \item 20\% parsing
    \pause
    \item 30\% testing of the methods
  \end{itemize}
\end{frame}

\section{Results}
\begin{frame}
  \frametitle{Results as statistics}
  \begin{itemize}
    \item 10 examples for training, 10 for validation
    \pause
    \item The automatic solver manages to prove 8 of each
    \pause
    \item The remaining examples seem to be unsolvable using Wilf-Zeilberger's method
  \end{itemize}
\end{frame}
\begin{frame}
  \frametitle{Results as an example}
  \begin{columns}[c]

  \column{.15\textwidth} % Left column and width

  \onslide<2->{
  \footnotesize
  $k<1$ and $k>n+1$ gives that $$\binom{n}{k-1}=0,$$ $$\implies$$ $$\lim_{k\to\pm\infty} G(n,k)=0.$$
  }
  \column{.5\textwidth} % Right column and width
  \begin{figure}
  \includegraphics[width=0.7\linewidth]{images/proof02.png}
  \end{figure}
  \column{.35\textwidth}
  \onslide<3->{
  \footnotesize
  For $n=0$ we get
  \begin{equation*}
    \begin{split}
      \sum_k (-1)^k\binom{n}{k}\binom{2k}{k}4^{n-k}&= \\
      (-1)^0\binom{0}{0}\binom{0}{0}4^0 &=1 \\
    \end{split}
  \end{equation*}
  Also $\binom{0}{0}=1$.
  }
  \end{columns}
\end{frame}

\section{Discussion and conclusions}
\begin{frame}
  \begin{columns}[t]

  \column{.5\textwidth} % Left column and width
  \center
  \ubf{DOES NOT WORK}

  \justify
  \begin{itemize}
    \item<2-> Cannot come up with solution, only prove
    \item<3-> Some parts are left for the user
    \item<4-> Similar examples with different results
  \end{itemize}
  \column{.5\textwidth} % Right column and width
  \center
  \ubf{WORKS WELL}

  \justify
  \begin{itemize}
    \item<5-> Solves most examples
    \item<6-> Gives a solution quickly (in seconds)
  \end{itemize}
  \end{columns}
\end{frame}
\begin{frame}
  \frametitle{Future work}
  \pause
  \begin{itemize}
    \item Can Wilf-Zeilberger's method be used in other fields? (not binomial coefficients)
    \pause
    \item Combine the program with guessing solution to identity
    \pause
    \item Computer algebra in general
  \end{itemize}
\end{frame}
\begin{frame}
  \Huge\center
  Thank you for listening!
\end{frame}

\appendix
\section*{Extra slides}%Extra slides
%Polynomials
\begin{frame}
  \frametitle{Polynomials -- Representation}
  Polynomial $$p(k)=a_0+a_1k+\ldots+a_mk^m$$ is stored as $$[a_0,a_1,\ldots,a_m].$$
\end{frame}
\begin{frame}
  \frametitle{Polynomials 1 -- Example}
  The polynomial $$p(k,m)=1+k^2+km-m^2+km^2+k^2m^2$$ is stored as $$\Big[[1,0,1],[0,1,0],[-1,1,1]\Big].$$
\end{frame}
\begin{frame}
  \frametitle{Polynomials -- Addition}
  Assume we want to add $f=[f_0,\ldots,f_{m_f}]$ and $g=[g_0,\ldots,g_{m_g}]$.\pause Then we get $h=[h_0,\ldots,h_m]$ where $m=max(m_f,m_g)$.\pause Then we have that
  $$h_i=f_i+g_i,$$
  if $f_i$ and $g_i$ are of one and the same variable.\pause Otherwise we get
  $$h_i=ADD(f_i,g_i).$$
\end{frame}
\begin{frame}
  \frametitle{Polynomials -- Multiplication}
  Assume we want to add $f=[f_0,\ldots,f_{m_f}]$ and $g=[g_0,\ldots,g_{m_g}]$.\pause Then we get $h=[h_0,\ldots,h_m]$ where $m=m_f+m_g$.\pause Then we have that
  $$h_i=\sum_{k=0}^i f_k\cdot g_{k-i},$$
  if $f_i$ and $g_i$ are of one and the same variable.\pause Otherwise we get
  $$h_i=\sum_{k=0}^i MULTIPLY(f_k,g_{k-i}).$$
\end{frame}
\begin{frame}
  \frametitle{Polynomials -- Division}
  Usually division ($a$ divided by $b$) is done by finding polynomials $q,r$ such that $$a=q\cdot b + r,$$
  and $deg(r)<deg(b)$.\pause This is not possible in integer coefficients.\pause Therefore we use $q,r,f$ such that $$f\cdot a=q\cdot b + r,$$ $deg(r)<deg(b)$ and $f$ has the same variable setup as the coefficients of $a$ and $b$.
\end{frame}
\begin{frame}
  \frametitle{Polynomials -- GCD}
\end{frame}
%Proof generation
\begin{frame}
  \frametitle{Proof generation}
  \ubf{Steps of proof generation}
  \pause
  \begin{itemize}
    \item Get input and parser
    \pause
    \item Parse input $\rightarrow$ get $F(n,k)$ and $\frac{a_k}{a_{k-1}}$
    \pause
    \item Get $G(n,k)$ from Gosper's algorithm
    \pause
    \item Write proof in \LaTeX format
    \pause
    \item Highlight parts that the user need to complete
  \end{itemize}
\end{frame}
%Dependencies of programs
\begin{frame}
  \frametitle{Dependencies of the code}
  \begin{figure}
  \includegraphics[width=0.6\linewidth]{images/dependency_graph_2.png}
  \end{figure}
\end{frame}

\iffalse
%------------------------------------------------
\section{First Section} % Sections can be created in order to organize your presentation into discrete blocks, all sections and subsections are automatically printed in the table of contents as an overview of the talk
%------------------------------------------------

\subsection{Subsection Example} % A subsection can be created just before a set of slides with a common theme to further break down your presentation into chunks

\begin{frame}
\frametitle{Paragraphs of Text}
Sed iaculis dapibus gravida. Morbi sed tortor erat, nec interdum arcu. Sed id lorem lectus. Quisque viverra augue id sem ornare non aliquam nibh tristique. Aenean in ligula nisl. Nulla sed tellus ipsum. Donec vestibulum ligula non lorem vulputate fermentum accumsan neque mollis.\\~\\

Sed diam enim, sagittis nec condimentum sit amet, ullamcorper sit amet libero. Aliquam vel dui orci, a porta odio. Nullam id suscipit ipsum. Aenean lobortis commodo sem, ut commodo leo gravida vitae. Pellentesque vehicula ante iaculis arcu pretium rutrum eget sit amet purus. Integer ornare nulla quis neque ultrices lobortis. Vestibulum ultrices tincidunt libero, quis commodo erat ullamcorper id.
\end{frame}

%------------------------------------------------

\begin{frame}
\frametitle{Bullet Points}
\begin{itemize}
\item Lorem ipsum dolor sit amet, consectetur adipiscing elit
\item Aliquam blandit faucibus nisi, sit amet dapibus enim tempus eu
\item Nulla commodo, erat quis gravida posuere, elit lacus lobortis est, quis porttitor odio mauris at libero
\item Nam cursus est eget velit posuere pellentesque
\item Vestibulum faucibus velit a augue condimentum quis convallis nulla gravida
\end{itemize}
\end{frame}

%------------------------------------------------

\begin{frame}
\frametitle{Blocks of Highlighted Text}
\begin{block}{Block 1}
Lorem ipsum dolor sit amet, consectetur adipiscing elit. Integer lectus nisl, ultricies in feugiat rutrum, porttitor sit amet augue. Aliquam ut tortor mauris. Sed volutpat ante purus, quis accumsan dolor.
\end{block}

\pause
\begin{block}{Block 2}
Pellentesque sed tellus purus. Class aptent taciti sociosqu ad litora torquent per conubia nostra, per inceptos himenaeos. Vestibulum quis magna at risus dictum tempor eu vitae velit.
\end{block}

\begin{block}{Block 3}
Suspendisse tincidunt sagittis gravida. Curabitur condimentum, enim sed venenatis rutrum, ipsum neque consectetur orci, sed blandit justo nisi ac lacus.
\end{block}
\end{frame}

%------------------------------------------------

\begin{frame}
\frametitle{Multiple Columns}
\begin{columns}[c]

\column{.45\textwidth} % Left column and width
\textbf{Heading}
\begin{enumerate}
\item Statement
\item Explanation
\item Example
\end{enumerate}

\column{.5\textwidth} % Right column and width
Lorem ipsum dolor sit amet, consectetur adipiscing elit. Integer lectus nisl, ultricies in feugiat rutrum, porttitor sit amet augue. Aliquam ut tortor mauris. Sed volutpat ante purus, quis accumsan dolor.

\end{columns}
\end{frame}

%------------------------------------------------
\section{Second Section}
%------------------------------------------------

\begin{frame}
\frametitle{Table}
\begin{table}
\begin{tabular}{l l l}
\toprule
\textbf{Treatments} & \textbf{Response 1} & \textbf{Response 2}\\
\midrule
Treatment 1 & 0.0003262 & 0.562 \\
Treatment 2 & 0.0015681 & 0.910 \\
Treatment 3 & 0.0009271 & 0.296 \\
\bottomrule
\end{tabular}
\caption{Table caption}
\end{table}
\end{frame}

%------------------------------------------------

\begin{frame}
\frametitle{Theorem}
\begin{theorem}[Mass--energy equivalence]
$E = mc^2$
\end{theorem}
\end{frame}

%------------------------------------------------

\begin{frame}[fragile] % Need to use the fragile option when verbatim is used in the slide
\frametitle{Verbatim}
\begin{example}[Theorem Slide Code]
\begin{verbatim}
\begin{frame}
\frametitle{Theorem}
\begin{theorem}[Mass--energy equivalence]
$E = mc^2$
\end{theorem}
\end{frame}\end{verbatim}
\end{example}
\end{frame}

%------------------------------------------------

\begin{frame}
\frametitle{Figure}
Uncomment the code on this slide to include your own image from the same directory as the template .TeX file.
%\begin{figure}
%\includegraphics[width=0.8\linewidth]{test}
%\end{figure}
\end{frame}

%------------------------------------------------

\begin{frame}[fragile] % Need to use the fragile option when verbatim is used in the slide
\frametitle{Citation}
An example of the \verb|\cite| command to cite within the presentation:\\~

This statement requires citation \cite{p1}.
\end{frame}

%------------------------------------------------

\begin{frame}
\frametitle{References}
\footnotesize{
\begin{thebibliography}{99} % Beamer does not support BibTeX so references must be inserted manually as below
\bibitem[Smith, 2012]{p1} John Smith (2012)
\newblock Title of the publication
\newblock \emph{Journal Name} 12(3), 45 -- 678.
\end{thebibliography}
}
\end{frame}

%------------------------------------------------

\begin{frame}
\Huge{\centerline{The End}}
\end{frame}
\fi
%----------------------------------------------------------------------------------------

\end{document}
