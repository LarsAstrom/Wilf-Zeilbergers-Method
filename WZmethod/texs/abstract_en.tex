The purpose of the thesis is to get a better understanding of computer algebra in general, and polynomial computer algebra in particular. This is done by implementing a library for a polynomial structure and methods that are needed to be able to perform operations on the polynomials. Then this library is used to implement Wilf-Zeilberger's method, which is a method that can be used to prove identities involving summation. The thesis consists mostly of three parts; theory, implementation and examples. In the theory section, all the theoretical results used in the project are presented. The implementation section then treats the difficulties arising when turning theory into practice, and focuses in particular on when theoretically easy methods and concepts become much more challenging in implementation. The program that is developed seems to work well, both on examples that were used throughout the project as testing and on validation examples that were found after all the implementation was done. This means that the program can solve and produce a paper proving that identities indeed are true. Therefore the thesis shows one example of how automated proofs can be generated, but mostly the thesis highlights the difficulties arising in computer algebra while implementing the specific example of Wilf-Zeilberger's method.
