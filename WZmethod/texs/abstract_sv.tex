Syftet med uppsatsen är att få en djupare och bättre förståelse för datoralgebra i allmänhet, och polynomisk datoralgebra i synnerhet. Detta görs genom att implementera ett bibliotek för polynom och metoder som behövs för att kunna utföra operationer på polynom. Därefter kommer detta bibliotek att användas för att implementera Wilf-Zeilbergers metod, som är en metod som kan användas för att bevisa vissa kombinatoriska identiteter som involverar summation. Uppsatsen består i huvudsak av tre delar; teori, implementation och exempel. I teoriavsnittet presenteras alla teoretiska resultat som används i projektet. Implementationsavsnittet behandlar därefter svårigheter som uppstår när teorin ska omvandlas till praktik och fokuserar framför allt på när teoretiskt enkla metoder och koncept blir mycket mer komplicerade och utmanande vid implementering. Programmet som har tagits fram verkar fungera väl, både på exempel som har använts genom hela projektet som testning och på valideringsexempel som hittades först efter att implementering var klar. Detta innebär att programmet löser problemen och producerar ett bevis för att identiteterna som ges faktiskt gäller. Därför visar uppsatsen ett exempel på hur automatiserade bevis kan genereras, men framför allt belyser uppsatsen svårigheterna som uppstår inom datoralgebra medan implementationen av det specifika exemplet Wilf-Zeilbergers metod görs.
