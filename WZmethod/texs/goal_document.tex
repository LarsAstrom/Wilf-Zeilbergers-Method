\documentclass[a4paper, 12pt]{article}
\setcounter{secnumdepth}{2}
\setcounter{tocdepth}{2}

%% Språk och font
\usepackage[english]{babel}
%\usepackage[utf8x]{inputenc}
\usepackage[T1]{fontenc}
\usepackage[utf8]{inputenc}
\usepackage{biblatex}
\usepackage{dirtytalk}

%% Sätter pappersstorlek och marginaler
\usepackage[a4paper,top=2.5cm,bottom=2.5cm,left=2.5cm,right=2.5cm,marginparwidth=1.75cm]{geometry}

%% Gör att figurer namnges efter kapitel
\usepackage{chngcntr}
\counterwithin{figure}{section}

\iffalse
%% Gör att nytt kapitel => ny sida
\usepackage{titlesec}
\newcommand{\sectionbreak}{\clearpage}
\let\stdsection\section
\renewcommand\section{\newpage\stdsection}
\fi

%% Användbara paket
\usepackage{amsmath}
\usepackage{amssymb}
\usepackage{graphicx}
\usepackage{subcaption}
\usepackage{caption}
\usepackage{float}
\usepackage[colorinlistoftodos]{todonotes}
\usepackage[colorlinks=true, allcolors=black]{hyperref}
\usepackage{enumerate}
\usepackage{amsthm}
\usepackage{textcomp}
\usepackage{gensymb}
\usepackage{csquotes}
\let\micro\micro
\let\perthousand\perthousand
\usepackage{listings}
\lstset{
  basicstyle=\ttfamily,
  columns=fullflexible,
  frame=single,
  breaklines=true,
  postbreak=\mbox{\textcolor{red}{$\hookrightarrow$}\space},
}

%% Sats-environment
\theoremstyle{definition}
\newtheorem{exmp}{Exempel}[section]
\newtheorem*{lsn}{Lösning}
\newtheorem{sats}{Sats}[section]
\newtheorem{definition}{Definition}[section]
\newtheorem*{anm}{Anmärkning}
\newtheorem{uppg}{}[subsection]

%% Figurer
\usepackage{siunitx}
\setlength{\parindent}{0pt}
\setlength{\parskip}{10pt}
\setlength{\fboxsep}{.5\fboxsep}
\newcommand{\mrel}{\mathrel{\bigcirc}}

%% För tabeller
\usepackage{array}
\newcolumntype{L}[1]{>{\raggedright\let\newline\\\arraybackslash\hspace{0pt}}m{#1}}
\newcolumntype{C}[1]{>{\centering\let\newline\\\arraybackslash\hspace{0pt}}m{#1}}
\newcolumntype{R}[1]{>{\raggedleft\let\newline\\\arraybackslash\hspace{0pt}}m{#1}}

%% Titel och författare
\title{Goal document.\\Working title: Computer algebraic implementation of Wilf Zeilberger's method}
\author{Lars Åström\\tpi15las@student.lu.se\\+46734304132\\Supervisor: Victor Ufnarovski}
\date{Preliminary dates:\\2019-09-01 -- 2019-12-20} %För inget datum

%\linespread{1.25}, for increased spacing

\begin{document}
\maketitle
\section{Background}
Showing equalities with computers have long been an interesting task. Solving equations is generally a fairly well known task, both analytically and numericly. One thing that is far harder to prove is showing equalities that involve infinite sums. A way to do this is by implementing Wilf-Zeilberger's method, which uses a clever change of variables to show identities which involve infintie sums. Two examples of identities that can be solved are
\begin{equation}\label{Identity 1}
  \sum_{i=0}^n \binom{n}{k}=2^n
\end{equation}
and
\begin{equation}\label{Identity 2}
  \sum_{k=0}^n (-1)^k\binom{n}{k}\binom{2k}{k}4^{n-k}=\binom{2n}{n}.
\end{equation}
The identity in \ref{Identity 1} can be shown beautifly using combinatorical arguments but the identity in \ref{Identity 2} is significantly harder. Both identities can however be shown in a structured manner using the WZ-method.

\section{Aims and problem statement}
The goal of the thesis is to create an open source software library that can use the WZ-method to show identities, like the ones in \ref{Identity 1} and \ref{Identity 2} but also many others. Although there already are implementations of the method in some mathematical languages, like Maple and Mathematica, there are -- to the author's and supervisor's knowledge -- no open source implementations of WZ-method. Furthermore another aim of the thesis is to get a better understanding of the problems involved when implementing computer algebra, like using polynomials and functions on them.

\section{Methods}
The library that the thesis will produce will be implemented in Python. In order to ease the implementation tools for operating with polynomials of many variables will be implemented. Thereafter a library for showing identities using Wilf-Zeilberger's method will be implemented, building on the polynomial tools.

\section{References to previous related works}
Wilf-Zeilberger's method is described in detail in Generating Functionology by Wilf (1990) -- see \url{https://www.math.upenn.edu/~wilf/gfology2.pdf}. One very important step of Wilf-Zeilberger's method is to do the change of variables. In order to find this, Gosper's algorithm is used. This algorithm is described in ''Decision procedure for indefinite hypergeometric summation'' by Gosper (1978), which can be found here: \url{https://www.pnas.org/content/pnas/75/1/40.full.pdf}.

\end{document}

\iffalse
%% För tre figurer med samma siffra.
\begin{figure}[H]
\centering
\begin{subfigure}[b]{.3\textwidth}
\centering
\includegraphics[width=.9\hsize]{AG1RatLinje_HittaM1lsn.png}
\caption{}\label{Fig: AG1RatLinje HittaM1lsn}
\end{subfigure}
\begin{subfigure}[b]{.3\textwidth}
\centering
\includegraphics[width=.9\hsize]{AG1RatLinje_HittaM2lsn.png}
\caption{}
\end{subfigure}
\begin{subfigure}[b]{.3\textwidth}
\centering
\includegraphics[width=.9\hsize]{AG1RatLinje_HittaM3lsn.png}
\caption{}\label{Fig: AG1RatLinje HittaM3lsn}
\end{subfigure}
\caption{}
\end{figure}

%% För två figurer med olika siffror.
\begin{figure}[H]
\centering
\begin{minipage}{.5\textwidth}
\centering
\includegraphics[width=.4\linewidth]{image1}
\captionof{figure}{A figure}
\label{fig:test1}
\end{minipage}
\begin{minipage}{.5\textwidth}
\centering
\includegraphics[width=.4\linewidth]{image1}
\captionof{figure}{Another figure}
\label{fig:test2}
\end{minipage}
\end{figure}

%% För en figur.
\begin{figure}[H]
\centering
\includegraphics[width=0.6\textwidth]{xxx.png}
\caption{}\label{}
\end{figure}

%% För en figur med 3*2 subfigurer
\begin{figure}[t!] % "[t!]" placement specifier just for this example
\begin{subfigure}{0.48\textwidth}
\includegraphics[width=\linewidth]{pic1.pdf}
\caption{First subfigure} \label{fig:a}
\end{subfigure}\hspace*{\fill}
\begin{subfigure}{0.48\textwidth}
\includegraphics[width=\linewidth]{pic2.pdf}
\caption{Second subfigure} \label{fig:b}
\end{subfigure}

\medskip
\begin{subfigure}{0.48\textwidth}
\includegraphics[width=\linewidth]{pic3.pdf}
\caption{Third subfigure} \label{fig:c}
\end{subfigure}\hspace*{\fill}
\begin{subfigure}{0.48\textwidth}
\includegraphics[width=\linewidth]{pic4.pdf}
\caption{Fourth subfigure} \label{fig:d}
\end{subfigure}

\medskip
\begin{subfigure}{0.48\textwidth}
\includegraphics[width=\linewidth]{pic5.pdf}
\caption{Fifth subfigure} \label{fig:e}
\end{subfigure}\hspace*{\fill}
\begin{subfigure}{0.48\textwidth}
\includegraphics[width=\linewidth]{pic6.pdf}
\caption{Sixth subfigure} \label{fig:f}
\end{subfigure}

\caption{My complicated figure} \label{fig:1}
\end{figure}

%%För att göra matriser:
\[
\begin{bmatrix}
    x_{11} & x_{12} & x_{13} & \dots  & x_{1n} \\
    x_{21} & x_{22} & x_{23} & \dots  & x_{2n} \\
    \vdots & \vdots & \vdots & \ddots & \vdots \\
    x_{d1} & x_{d2} & x_{d3} & \dots  & x_{dn}
\end{bmatrix}
\]

%Tabell
\begin{center}
        \begin{tabular}{c|c|c|c}
            Data set & Base 1 & Base 2 & Base 3 \\ \hline
            Training data & 821 & 860 & 945 \\ \hline
            Test data & 795 & 649 & 697 \\ \hline
        \end{tabular}
    \end{center}
\fi
