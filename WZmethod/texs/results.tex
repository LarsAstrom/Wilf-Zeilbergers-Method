Throughout the thesis some identities have been used for testing the code, these will be called training examples. All identities that have been used in this thesis have been collected from \reference{gould}, \reference{bucht} and \reference{tesler}. The identities used as training examples are the following:
\begin{enumerate}
  \item $\sum_{k=0}^n \binom{n}{k} = 2^n$ \label{ID: 1}
  \item $\sum_{k=0}^n (-1)^k\binom{n}{k}\binom{2k}{k}4^{n-k}=\binom{2n}{n}$ \label{ID: 2}
  \item $\sum_{k=0}^n \binom{n}{k}^2 = \binom{2n}{n}$ \label{ID: 3}
  \item $\sum_{k=0}^n 2^k\binom{n}{k} = 3^n$ \label{ID: 4}
  \item $\sum_{k=0}^n k\binom{n}{k} = n2^{n-1}$ \label{ID: 5}
  \item $\sum_{k=1}^n \frac{1}{k(k-1)} = 1-\frac{1}{n}$ \label{ID: 6}
  \item $\sum_{k=0}^n \binom{k}{c} = \binom{n+1}{c+1}$ \label{ID: 7}
  \item $\sum_{k=0}^n \binom{r+k}{k} = \binom{r+n+1}{n}$ \label{ID: 8}
  \item $\sum_{k=0}^n \binom{m-k}{n-k} = \binom{m+1}{n}$ \label{ID: 9}
  \item $\sum_{k=n}^\infty \frac{1}{\binom{k}{n}}=\frac{n}{n-1}$ \label{ID: 10}
\end{enumerate}
Some of these identities are quite trivial to show with combinatorial arguments, but others are not as easy. For instance identity \ref{ID: 1} comes from counting in how many ways we can choose objects out of $n$ different objects, while identity \ref{ID: 2} is not very easy to prove by providing a combinatorial argument. When these identities are shown, a \LaTeX file is produced. The proofs of all identities are provided in chapter \ref{Ch: Attachments}. As we can see the identities \ref{ID: 7} and \ref{ID: 8} cannot be shown, the reason for this is that the ''get f'' method cannot give a solution, since $f$ cannot be a polynomial. This is because the degree of $f$ depends on $c$ and $r$, respectively, and therefore Wilf-Zeilberger's method does not work well on the problem.

After the program worked for all the training examples the program was validated on other identities (called validation examples), in the hopes that the program works on identities it has not been developed to work on as well. The tests that were used are:
\begin{enumerate}
  \setcounter{enumi}{10}
  \item $\sum_{k=0}^n 3^k\binom{n}{k} = 4^n$ \label{ID: 11}
  \item $\sum_{k=0}^n 4^k\binom{n}{k} = 5^n$ \label{ID: 12}
  \item $\sum_{k=0}^n \binom{n}{2k} = 2^{n-1}$ \label{ID: 13}
  \item $\sum_{k=0}^n \binom{n}{k}\binom{k}{b} = \binom{n}{b}$ \label{ID: 14}
  \item $\sum_{k=0}^\infty 2^{-k}\binom{n+k}{k} = 2^{n+1}$ \label{ID: 15}
  \item $\sum_{k=0}^n \frac{\binom{n}{k}}{\binom{2n-1}{k}} = 2$ \label{ID: 16}
  \item $\sum_{k=0}^n k\frac{\binom{n}{k}}{\binom{2n-1}{k}} = 2\frac{n}{n+1}$ \label{ID: 17}
\end{enumerate}
As we can see in chapter \ref{Ch: Attachments} the program manages to solve most of the validation examples. The program solves identities \ref{ID: 11}, \ref{ID: 12} and \ref{ID: 15} perfectly. For identities \ref{ID: 16} and \ref{ID: 17} the program manages to get a $G$ which works in Wilf-Zeilberger's method but has not been able to verify that the first condition for a certifying pair is fulfilled. This has been done by hand afterwards though. For identities \ref{ID: 13} and \ref{ID: 14} the program does not manage to prove the identity, this is since it cannot find a $f$ that works. The reason for this is similar to why the program could not solve identities \ref{ID: 7} and \ref{ID: 8}, and thus it was expected.
