\documentclass[a4paper]{article}
\usepackage[T1]{fontenc}        % För svenska bokstäver
%\usepackage[swedish]{babel}    %Svenska skrivregler och rubriker
%\usepackage[dvips]{graphics}
\usepackage{epsfig}


\title{A title}
\author{X Jobbare}
%\date{}

\def\svd {Singular Value Decomposition}

\begin{document}
\maketitle

\begin{abstract}
Här skriver jag in mitt abstract.
\end{abstract}

\pagebreak
.
\pagebreak

\tableofcontents

\pagebreak
.
\pagebreak

\section{Introduction}

Allmän problemformulering \svd lakjsfdl.

Previous work

Overview of the thesis
In chapter 2 some background material is ...


\subsection{Background}

\subsection{Aim of the thesis}

% Here is a figure
\begin{figure}
\begin{center}
\epsfig{figure=figs/trafik.eps,width=5cm}
% lägg till figurbilden
\caption{The figure illustrates ...}
\label{f_traffic}
\end{center}
\end{figure}


As can be seen in Figure~\ref{f_traffic} the xxx.

% Here is a table
\begin{table}[htbp]
  \begin{center}
    \begin{tabular}{|l|r|}
      \hline
      Cars & 0.4178 \\
      \hline
      Multiple cars & 0.0375 \\
      Pedestrians & 0.1684 \\
      Multiple Pedestrians & 0.1247 \\
      Bicycles & 0.1004 \\
      Multiple Bicycles & 0.0081 \\
      Multiple objects of mixed classes & 0.0314 \\
      Others & 0.1116 \\
      \hline
    \end{tabular}
    \caption{The distribution of tracks of different classes.}
    \label{t_class_hist}
  \end{center}
\end{table}

% This is an equation without equation numbering
$$ f(Z=z \, | \, K = k) = \frac{1}{N} \sum_{k=1}^N N(o_i,\sigma_i)
\enspace .$$

Here $z = (x,y) = \int_0^5 e^x dx$ denotes the position of an object, $k$ denotes its
class. $N(o,\sigma$ is the propability density function of the normal
distributin.

It now follows that
$$z = (x,y) = \int_0^5 e^x dx, $$
where $z$ denotes temperature and
$$z = (x,y) = \int_0^5 e^x dx. $$

Similar to the above we have
\begin{equation}
  z = (x,y) = \int_0^5 e^x dx .
  \label{minekvation}
  \end{equation}

According to  (\ref{minekvation}) this xxx.

\subsection{Related work}

Several methods for estimating and updating the background images have
been used.  One such method \cite{stauffer-ccvpr-99} is based on
modeling the probability distribution of each background pixel as a
mixture of Gaussians using the EM algorithm.  The dominant component
is considered to be the background and is used for estimating
foreground pixels.

\section{Methods}

\section{Geometry}

\subsection{Homogenous coordinates}
A straightforward way of representing a point on a plane is by
using two numbers \((x,y)\). This notation is known as euclidian
coordinates and the meaning of such a coordinate pair is probably
familiar to everyone and requires no further explanation. It turns
out however that a slightly different representation of points
provide great simplifications to the notation of many relations in
projective geometry. F.g. a line on a plane can be represented by
three numbers \((a,b,c)\) such that a point \((x,y)\) on the line
satisfies the equation \(ax+by+c=0\). If instead of \((x,y)\) we
represent the point by \((x,y,1)\), the equation can be written as
a product of two vectors: \(a,b,c)\cdot(x,y,1)^{\top}=0\). We also
see that multiplying \((x,y,1)\) by a scalar, \(\alpha\), will
give rise to a new representation \((\alpha x,\alpha y,\alpha)\)
which satisfies the equation just as well. In fact, using
homogenous coordinates, the two representations \((x,y,1)\) and
\(\alpha(x,y,1)\) are equivalent and represent the same point for
any non-zero scalar \(\alpha\). An arbitrary 3-vector \((x,y,w)\)
in homogenous coordinates represents the point \((x / w,y / w)\)
in euclidian coordinates.

\subsection{The camera equation}
A simple mathematical description of a projective camera is given
by the \emph{pinhole camera model}, shown in figure 2.1.

%\begin{figure}
%    \begin{center}
%        \includegraphics{pinhole.eps}
%    \end{center}
%    \caption{Pinhole camera model}
%\end{figure}
%\resizebox{!}{20mm}{\includegraphics{pinhole.eps}}

Som man kan se i \cite{alvarez-guichard-etal-crasp-92} så är.


\section{Results}

\section{Conclusions}

\section*{Acknowledgements}

\bibliographystyle{plain}
\bibliography{vision.newkeys}



\end{document}
