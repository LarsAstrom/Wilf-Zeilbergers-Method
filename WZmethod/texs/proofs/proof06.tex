\documentclass{article}
\usepackage[utf8]{inputenc}
\usepackage{amsmath}
\let\oldforall\forall
\renewcommand{\forall}{\hspace*{2mm}\oldforall\hspace*{1mm}}
\title{Proof}
\author{Automatic WZ-method prover}
\date{2019-11-26}
\begin{document}
\maketitle
We want to prove that
\begin{equation}\label{Eq: 1}
\sum \frac{1}{k(k-1)} = 1-\frac{1}{n}
\end{equation}
holds. By dividing equation \ref{Eq: 1} by the right hand side we get
\begin{equation}
F(n,k)=\frac{\frac{1}{k(k-1)}}{1-\frac{1}{n}}
\end{equation}
We use proof certificate
\begin{equation}
R(n,k)=\frac{k-2}{n^2},
\end{equation}
which is the same as using
\begin{equation}
G(n,k)=\frac{k-2}{n^2}\frac{\frac{1}{(k-1)((k-1)-1)}}{1-\frac{1}{n}},
\end{equation}
the automatic solver has  verified that
\begin{equation}\label{Eq: WZ1}
F(n+1,k)-F(n,k)=G(n,k+1)-G(n,k).
\end{equation}
Thereafter user now has to verify that
\begin{equation}
\lim_{k\to\pm\infty}G(n,k)=0\forall n.
\end{equation}
Then we get
\begin{equation}
\sum_k F(n+1,k)-F(n,k)=\sum_k G(n,k+1)-G(n,k)=0\end{equation}Lastly equation \ref{Eq: 1} needs to be verified for some $n$, for instance $n=0$. Thereafter the identity is shown.
\end{document}
