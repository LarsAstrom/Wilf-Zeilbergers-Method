\section{Wilf Zeilberger method}
Wilf-Zeilberger's method is a method to show an equality by using a certifying pair, or a proof certificate. The methodology in the proof  is that assume we want to prove %~\cite{gf}
\begin{equation}\label{Eq: Theory,PS}
  \sum_{k=-\infty}^\infty A(n,k) = B(n).
\end{equation}
Then we let
\begin{equation}
  F(n,k)=\frac{A(n,k)}{B(n)}
\end{equation}
Now equation \ref{Eq: Theory,PS} becomes
\begin{equation}\label{Eq: Theory,WZ}
  \sum_{k=-\infty}^\infty F(n,k) = 1
\end{equation}
The way we want to prove this is by first proving
\begin{equation}\label{Eq: Theory,WZ2}
  \sum_{k=-\infty}^\infty F(n+1,k)-F(n,k) = 0
\end{equation}
and then calculate the left hand side of \ref{Eq: Theory,WZ} for one $n$, usually but not always $n=0$. Then we get that
\begin{equation}
  \sum_{k=-\infty}^\infty F(n+1,k)=\sum_{k=-\infty}^\infty F(n,k)
\end{equation}
which in turns gives us that
\begin{equation}
  \sum_{k=-\infty}^\infty F(n,k) = c,
\end{equation}
where $c$ is a constant, for all values of $n$ and thus equal to the result of our previous calculations. The trick we use to prove equation \ref{Eq: Theory,WZ2} is to find another function $G(n,k)$ which satisfies:
\begin{enumerate}[i)]
  \item
  \begin{equation}\label{Eq: Theory,first condition}
    F(n+1,k)-F(n,k)=G(n,k+1)-G(n,k)
  \end{equation}
  \item
  \begin{equation}\label{Eq: Theory,second condition}
    \lim_{k\to\pm\infty}G(n,k)=0, \forall n
  \end{equation}
\end{enumerate}
If these two conditions are fulfilled, then equation \ref{Eq: Theory,WZ2} is fulfilled since by replacing $F(n+1,k)-F(n,k)$ with $G(n,k+1)-G(n,k)$ we get a telescopic sum:
\begin{equation*}
\begin{split}
\sum_{k=-\infty}^\infty F(n+1,k)-F(n,k) & = \sum_{k=-\infty}^\infty G(n,k+1)-G(n,k) = \\
 & = \lim_{M\to\infty} \sum_{k=-M}^M G(n,k+1)-G(n,k) = \\
 & = \lim_{M\to\infty} \big[G(n,M+1)-G(n,-M)\big] = 0.
\end{split}
\end{equation*}
If we summarize the method we have the following steps:
\begin{enumerate}
  \item Start with equation on the form $$\sum_{k=-\infty}^\infty A(n,k) = B(n).$$
  \item Let $$F(n,k)=\frac{A(n,k)}{B(n)}.$$
  \item Find $G(n,k)$ such that equations \ref{Eq: Theory,first condition} and \ref{Eq: Theory,second condition} are fulfilled.
  \item Show that $\sum_{k=-\infty}^\infty F(0,k)=1$.
\end{enumerate}
We can formalize the concepts in the method like this.
\begin{definition}
  A pair of functions $(F,G)$ that satisfy equations \ref{Eq: Theory,first condition} and \ref{Eq: Theory,second condition} are said to ''certify'' an identity like \ref{Eq: Theory,WZ} (or is simply called a ''certifying pair''). We can also speak of a ''proof certificate'' $R(n,k)$ of an identity. That is a function such that $G(n,k)=R(n,k)F(n,k-1)$ gives that $(F,G)$ is a certifying pair to the identity.
\end{definition}
Let us show an short example of how the method can work.
\begin{example}
  Show that $$\sum_{k=0}^n \binom{n}{k} = 2^n.$$
\end{example}
\begin{solution}
  We will use the method and do all the steps mentioned above.
  \begin{enumerate}
    \item We have $A(n,k)=\binom{n}{k}$ and $B(n)=2^n$.
    \item Let $$F(n,k)=\frac{A(n,k)}{B(n)}=\frac{\binom{n}{k}}{2^n}.$$
    \item With $$G(n,k)=-\frac{\binom{n}{k-1}}{2^{n+1}},$$ we have that
      \begin{equation*}
        \begin{split}
          F(n+1,k)-F(n,k) & = \frac{\binom{n+1}{k}}{2^{n+1}}-\frac{\binom{n}{k}}{2^{n}}= \\
          & = \frac{1}{2^{n+1}}\binom{n}{k}\Bigg(\frac{n+1}{n+1-k}-2\Bigg) = \\
          & = \frac{1}{2^{n+1}}\binom{n}{k}\Bigg(\frac{k}{n+1-k}-1\Bigg) = \\
          & = \frac{1}{2^{n+1}}\Bigg(\binom{n}{k-1}-\binom{n}{k}\Bigg) = \\
          & = G(n,k+1)-G(n,k)
        \end{split}
      \end{equation*}
      Furthermore, we have that $G(n,k) = 0$ when $k<0$ and $k>n$.
    \item Lastly we need to show that $$\sum_{k=0}^n \frac{\binom{n}{k}}{2^n}=1$$ for some $n$. If we choose $n=0$ we see that this equality holds. Therefore the equality holds for all $n$.
  \end{enumerate}
\end{solution}
Now that we have seen an example, hopefully the methodology is clear. What still is not clear is how to find the function $G(n,k)$. This is in general a hard task to do by hand, but is certainly managable by using a computer since there are algorithms -- for instance Gosper's algorithm -- that can derive $G(n,k)$ in certain cases. This implementation is what this thesis is contributing to do. %~\cite{gf}

\section{Gosper's algorithm}


\section{Polynomials}

\section{Division, modulo and GCD}
