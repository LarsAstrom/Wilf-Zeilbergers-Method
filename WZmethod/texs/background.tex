Computer algebra, also called symbolic computation or algebraic computation, is the study of algorithms and software for simplifying and manipulating mathematical expressions. Computer algebra has many applications, is a part of many different mathematical languages, such as Maple and Mathematica, and is used widely in many mathematical and engineering applications, such as physics and cryptography (see \reference{maple} and \reference{computeralgebra}).

In combinatorics binomial coefficients are constantly used for counting different things. Also the sum of binomial coefficients are often of interest in solving the theoretical questions in combinatorics and similar fields. Some combinatorial identities have beautiful and simple proofs using combinatorial arguments while others require long and complicated proofs. Therefore Wilf-Zeilberger's method comes in handy, to automize the proof procedure.

In the thesis Wilf-Zeilberger's method will be used and implemented, which is a method to prove the correctness of an (infinite) sum. One example where it can be used is to prove
\begin{equation}\label{Eq: Background}
  \sum_{k=0}^n (-1)^k\binom{n}{k}\binom{2k}{k}4^{n-k}=\binom{2n}{n}.
\end{equation}
Here we see that we have a finite sum, but it can also be seen as an infinite sum from $-\infty$ to $\infty$ too, if we define $\binom{n}{a}=0$, when $a<0$ or $a>n$.

The steps of the method will of course be shown in chapter \ref{Ch: Theory}, but the general idea is to change our addends in the sum to something that telescopes. After this is done, proving the identity gets reduced to prove the identity for a single $n$. As we see in equation \eqref{Eq: Background} this is quite easy to prove for some $n$, for instance $n=0$. Although it is not always easy to prove the identity for some $n$, it is usually the case.

Wilf-Zeilberger's method itself is not a very new method, it dates back to 1990 (see \reference{wz}). Even implementations of the method in computer algebraic softwares is not very new, for instance an implementation of the method in Mathematica was published in 1994 (see \reference{mathematica}). The contribution of this thesis is therefore not an implementation of something that has not been implemented before, but rather an open source version of it. Furthermore, as stated in chapter \ref{Ch: Introduction}, a large part of the goal is to get a better understanding of computer algebra.
