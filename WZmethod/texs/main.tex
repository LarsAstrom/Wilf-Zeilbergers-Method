\documentclass[a4paper, 12pt]{article}
\setcounter{secnumdepth}{2}
\setcounter{tocdepth}{2}

%% Språk och font
\usepackage[english]{babel}
%\usepackage[utf8x]{inputenc}
\usepackage[T1]{fontenc}
\usepackage[utf8]{inputenc}
\usepackage{biblatex}
\usepackage{dirtytalk}

%% Sätter pappersstorlek och marginaler
\usepackage[a4paper,top=2.5cm,bottom=2.5cm,left=2.5cm,right=2.5cm,marginparwidth=1.75cm]{geometry}

%% Gör att figurer namnges efter kapitel
\usepackage{chngcntr}
\counterwithin{figure}{section}

\iffalse
%% Gör att nytt kapitel => ny sida
\usepackage{titlesec}
\newcommand{\sectionbreak}{\clearpage}
\let\stdsection\section
\renewcommand\section{\newpage\stdsection}
\fi

%% Användbara paket
\usepackage{amsmath}
\usepackage{amssymb}
\usepackage{graphicx}
\usepackage{subcaption}
\usepackage{caption}
\usepackage{float}
\usepackage[colorinlistoftodos]{todonotes}
\usepackage[colorlinks=true, allcolors=black]{hyperref}
\usepackage{enumerate}
\usepackage{amsthm}
\usepackage{textcomp}
\usepackage{gensymb}
\usepackage{csquotes}
\let\micro\micro
\let\perthousand\perthousand
\usepackage{listings}
\lstset{
  basicstyle=\ttfamily,
  columns=fullflexible,
  frame=single,
  breaklines=true,
  postbreak=\mbox{\textcolor{red}{$\hookrightarrow$}\space},
}

%% Sats-environment
\theoremstyle{definition}
\newtheorem{exmp}{Exempel}[section]
\newtheorem*{lsn}{Lösning}
\newtheorem{sats}{Sats}[section]
\newtheorem{definition}{Definition}[section]
\newtheorem*{anm}{Anmärkning}
\newtheorem{uppg}{}[subsection]

%% Figurer
\usepackage{siunitx}
\setlength{\parindent}{0pt}
\setlength{\parskip}{10pt}
\setlength{\fboxsep}{.5\fboxsep}
\newcommand{\mrel}{\mathrel{\bigcirc}}

%% För tabeller
\usepackage{array}
\newcolumntype{L}[1]{>{\raggedright\let\newline\\\arraybackslash\hspace{0pt}}m{#1}}
\newcolumntype{C}[1]{>{\centering\let\newline\\\arraybackslash\hspace{0pt}}m{#1}}
\newcolumntype{R}[1]{>{\raggedleft\let\newline\\\arraybackslash\hspace{0pt}}m{#1}}

%% Titel och författare
\title{Template}
\author{Lars Åström}
%\date{} %För inget datum

%\linespread{1.25}, for increased spacing

\begin{document}
\maketitle

\end{document}

\iffalse
%% För tre figurer med samma siffra.
\begin{figure}[H]
\centering
\begin{subfigure}[b]{.3\textwidth}
\centering
\includegraphics[width=.9\hsize]{AG1RatLinje_HittaM1lsn.png}
\caption{}\label{Fig: AG1RatLinje HittaM1lsn}
\end{subfigure}
\begin{subfigure}[b]{.3\textwidth}
\centering
\includegraphics[width=.9\hsize]{AG1RatLinje_HittaM2lsn.png}
\caption{}
\end{subfigure}
\begin{subfigure}[b]{.3\textwidth}
\centering
\includegraphics[width=.9\hsize]{AG1RatLinje_HittaM3lsn.png}
\caption{}\label{Fig: AG1RatLinje HittaM3lsn}
\end{subfigure}
\caption{}
\end{figure}

%% För två figurer med olika siffror.
\begin{figure}[H]
\centering
\begin{minipage}{.5\textwidth}
\centering
\includegraphics[width=.4\linewidth]{image1}
\captionof{figure}{A figure}
\label{fig:test1}
\end{minipage}
\begin{minipage}{.5\textwidth}
\centering
\includegraphics[width=.4\linewidth]{image1}
\captionof{figure}{Another figure}
\label{fig:test2}
\end{minipage}
\end{figure}

%% För en figur.
\begin{figure}[H]
\centering
\includegraphics[width=0.6\textwidth]{xxx.png}
\caption{}\label{}
\end{figure}

%% För en figur med 3*2 subfigurer
\begin{figure}[t!] % "[t!]" placement specifier just for this example
\begin{subfigure}{0.48\textwidth}
\includegraphics[width=\linewidth]{pic1.pdf}
\caption{First subfigure} \label{fig:a}
\end{subfigure}\hspace*{\fill}
\begin{subfigure}{0.48\textwidth}
\includegraphics[width=\linewidth]{pic2.pdf}
\caption{Second subfigure} \label{fig:b}
\end{subfigure}

\medskip
\begin{subfigure}{0.48\textwidth}
\includegraphics[width=\linewidth]{pic3.pdf}
\caption{Third subfigure} \label{fig:c}
\end{subfigure}\hspace*{\fill}
\begin{subfigure}{0.48\textwidth}
\includegraphics[width=\linewidth]{pic4.pdf}
\caption{Fourth subfigure} \label{fig:d}
\end{subfigure}

\medskip
\begin{subfigure}{0.48\textwidth}
\includegraphics[width=\linewidth]{pic5.pdf}
\caption{Fifth subfigure} \label{fig:e}
\end{subfigure}\hspace*{\fill}
\begin{subfigure}{0.48\textwidth}
\includegraphics[width=\linewidth]{pic6.pdf}
\caption{Sixth subfigure} \label{fig:f}
\end{subfigure}

\caption{My complicated figure} \label{fig:1}
\end{figure}

%%För att göra matriser:
\[
\begin{bmatrix}
    x_{11} & x_{12} & x_{13} & \dots  & x_{1n} \\
    x_{21} & x_{22} & x_{23} & \dots  & x_{2n} \\
    \vdots & \vdots & \vdots & \ddots & \vdots \\
    x_{d1} & x_{d2} & x_{d3} & \dots  & x_{dn}
\end{bmatrix}
\]

%Tabell
\begin{center}
        \begin{tabular}{c|c|c|c}
            Data set & Base 1 & Base 2 & Base 3 \\ \hline
            Training data & 821 & 860 & 945 \\ \hline
            Test data & 795 & 649 & 697 \\ \hline
        \end{tabular}    
    \end{center}
\fi